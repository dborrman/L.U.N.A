\section{Conclusions}
\label{sec:conclusions}

This paper has presented a new cost efficient approach to 3D laser scanning: The L.U.N.A. sphere.
It uses a 2D laser scanner mounted inside a spherical robot and the inherent rotational movement to form a radial scanning pattern, hence creating a 3D point cloud.
The spherical robot is based on conversation of angular momentum and uses flywheels to drive the robot forward. 

The prototype developed for the tests in this paper was able to move in one direction reliably on soft surfaces (such as rubber), however had difficulties with slippage on hard and low friction surfaces.

With regards to 3D laser scanning this paper has delivered a proof of concept, even though the results leave room for improvement.
Needless to say, a lot of work still has to be done. 
In future work, we plan to overcome the biggest issues, i.e, the reflection of the laser beams by the exterior shell and synchronization issues between the IMU system and the laser scanner. 

This includes improving the field of view of the laser scanner and extending the robot to two dimensional movement control. This enables autonomous mapping of environments using the L.U.N.A. sphere. 