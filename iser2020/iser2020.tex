
%%%%%%%%%%%%%%%%%%%% author.tex %%%%%%%%%%%%%%%%%%%%%%%%%%%%%%%%%%%
%
% sample root file for your "contribution" to a contributed volume
%
% Use this file as a template for your own input.
%
%%%%%%%%%%%%%%%% Springer %%%%%%%%%%%%%%%%%%%%%%%%%%%%%%%%%%


% RECOMMENDED %%%%%%%%%%%%%%%%%%%%%%%%%%%%%%%%%%%%%%%%%%%%%%%%%%%
\documentclass[graybox]{svmult}

% choose options for [] as required from the list
% in the Reference Guide

\usepackage{mathptmx}       % selects Times Roman as basic font
\usepackage{helvet}         % selects Helvetica as sans-serif font
\usepackage{courier}        % selects Courier as typewriter font
\usepackage{type1cm}        % activate if the above 3 fonts are
                            % not available on your system
%
\usepackage{makeidx}         % allows index generation
\usepackage{graphicx}        % standard LaTeX graphics tool
                             % when including figure files
\usepackage{multicol}        % used for the two-column index
\usepackage[bottom]{footmisc}% places footnotes at page bottom
\usepackage{marginnote}
\newcommand{\TODO}[1]{\marginnote{TODO: #1}}    % TODO Befehl
\usepackage{units,subfigure,amsmath}
\usepackage{wrapfig}
\usepackage{todonotes}
\usepackage{hyperref}
\hypersetup{
    colorlinks=true,
    linkcolor=blue,
    filecolor=magenta,      
    urlcolor=cyan,
}
\usepackage{url}
% see the list of further useful packages
% in the Reference Guide

\makeindex             % used for the subject index
                       % please use the style svind.ist with
                       % your makeindex program

%\usepackage{cleveref}[2012/02/15]
%\crefformat{footnote}{#2\footnotemark[#1]#3}

\renewcommand{\O}{{\cal O}}
\renewcommand{\leadsto}{\rightsquigarrow}
\newcommand{\V}[1]{\text{\boldmath $#1$}}    % Format for "Vector"
\newcommand{\M}[1]{\V{#1}}                   % Format for "Matrix"

\newcommand{\R}{\mathbbm{R}}                 % set of real number
\newcommand{\N}{\mathbbm{N}}                 % set of natural numbers
\newcommand{\C}{\mathbbm{C}}                 % ...
\newcommand{\1}{\mathbbm{1}}                 % identity matrix


%%%%%%%%%%%%%%%%%%%%%%%%%%%%%%%%%%%%%%%%%%%%%%%%%%%%%%%%%%%%%%%%%%%%%%%%%%%%%%%%%%%%%%%%%

%%
% Motivation, Problem Statement, Related Work (one page)
% Technical Approach (one page)
% Results (one page)
% Experiments completed or scheduled (one page)
% Main experimental insights (one page)
% References (one page)
%%

\begin{document}

\title*{L.U.N.A. - A Laser-Mapping Unidirectional Navigation Actuator} 
\author{Jasper Zevering, Anton Bredenbeck, Fabian Arzberger,
  Dorit Borrmann and Andreas N\"uchter}
% FIXME sort the authors
% Use \authorrunning{Short Title} for an abbreviated version of
% your contribution title if the original one is too long
\institute{All authors are with Informatics VII -- Robotics and
  Telematics, University of W\"urzburg, Am Hubland, 97074 W\"urzburg
  \email{borrmann@informatik.uni-wuerzburg.de} \\| \email{fabian.arzberger@stud-mail.uni-wuerzburg.de}\\ | \email{anton.bredenbeck@stud-mail.uni-wuerzburg.de}\\ | \email{jasper.zevering@stud-mail.uni-wuerzburg.de} 
%\and Name of Second Author \at Name, Address of Institute 
%\email{name@email.address}
}
%
% Use the package "url.sty" to avoid
% problems with special characters
% used in your e-mail or web address
%
\maketitle

%\abstract*{Each chapter should be preceded by an abstract (10--15 lines long) 
%that summarizes the content. The abstract will appear \textit{online} at 
%\url{www.SpringerLink.com} and be available with unrestricted access. This 
%allows unregistered users to read the abstract as a teaser for the complete 
%chapter. As a general rule the abstracts will not appear in the printed 
%version 
%of your book unless it is the style of your particular book or that of the 
%series to which your book belongs.
%Please use the 'starred' version of the new Springer \texttt{abstract} command 
%for typesetting the text of the online abstracts (cf. source file of this 
%chapter template \texttt{abstract}) and include them with the source files of 
%your manuscript. Use the plain \texttt{abstract} command if the abstract is 
%also to appear in the printed version of the book.}
\abstract{
The abstract goes here.
}

\section{Introduction}
\label{sec:introduction}

In today's world, autonomous robots have found their way into everyday life in a variety of ways. This includes, but isn't limited to, the vacuum cleaner that independently navigates one's living-room or mobile robots employed for exploration of areas that are too dangerous for humans. To foster new advances in the latter, specifically for underground environments, the  Defense Advanced Research Projects Agency (DARPA) of the US Defense Department established the yearly "SubT" Challenge in 2017. In this challenge, teams are tasked to "Drive novel approaches and technologies to allow warfighters and first-responders to rapidly map, navigate, and search dynamic underground environments."  \cite{allen} proving the demand for further research in this domain. One difficulty of this challenge is building an accurate 3D model of the environment, i.e. mapping the surroundings. The teams that participate in the DARPA challenge take advantage of high-quality hardware, such as state-of-the-art 3D laser-scanners and cameras, thus making their solutions rather expensive. However, the demand for mapping-solutions in the low-cost sector is non-negligible. 

\todo[inline]{Previous low cost 3D laser scanning approaches.I'm unhappy with the complete next paragraph. From here...}
"Classical Mechanics Scanner" \cite{classical_mechanics_scanner}

Previous work was also done at the Julius-Maximilians University W\"urzburg \cite{ISER2018}. The RADLER (RADial LasER scanning device) consists of a 2D laser scanner attached to the axle of a unicycle. An operator then pushes the unicycle along a requested path. The inherent rotation of the wheel then creates a radial 3D laser-scanning pattern. However, this approach still requires an operator, therefore does not fulfill the autonomy requirements. 

A more autonomous approach was taken in \cite{3D_per_2D_based}. The authors mounted a rotating 2D laser-scanner on top of a \href{https://www.turtlebot.com}{turtle-bot} thus removing the need of an operator. In contrast to the RADLER however, does the turtle-bot not provide an inherent rotation. Therefore an extra actuator is required to create the radial 3D scanning-pattern. \todo[inline]{...to here}

Building upon the results of the RADLER  this paper presents a novel approach to low-cost 3D laser-scanning using a 2D laser-scanner inside a torque driven spherical robot: the L.U.N.A. - sphere (Laser-mapping Unidirectional Navigation Actuator). The 2D laser-scanner is fixed to the spherical structure, hence a similar situation as with the RADLER is given: the inherent rotation of the sphere creates a radial 3D scanning pattern. 


\section{Technical approach}

\subsection{Hardware Setup}

\subsection{Sensor Integration}


\section{Conclusions}
\label{sec:conclusions}

In this paper a new cost efficient approach to 3D lase scanning was proposed: The L.U.N.A. sphere. It uses a 2D laser scanner mounted inside a spherical robot and uses the inherent rotational movement to form a radial scanning pattern and hence create a 3D point cloud. The spherical robot is based on conversation of angular momentum and uses flywheels to drive the robot forward. 

The prototype developed for the tests in this paper was able to move in one direction reliably on soft surfaces (such as rubber), however had difficulties with slippage on hard and slippery surfaces. In regards of 3D scanning this paper delivered a proof of concept, even though the result remain unsatisfactory as of right now. The biggest issues to overcome are reflections of the laser scanner beams by the exterior shell and synchronization issues between the IMU system and the laser scanner. 

Before the application of such a robot is possible more work is required. This could include improving the field of view of the laser scanner and extending the robot to two dimensional movement control. This would then enable autonomous mapping of environments using the L.U.N.A. sphere. 


\begin{acknowledgement}
The authors thank Dieter Ziegler, Dorit and Andreas
\end{acknowledgement}

\bibliographystyle{plain}
\bibliography{andreas_publications}

\end{document}
