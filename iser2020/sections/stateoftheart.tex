\section{State of the Art}
\label{sec:stateOfTheArt}
As evolution of RADLER as hand-driven radial scanner device, a self-driving spherical approach was chosen to ensure robustness and autonomy.  There have been several works in literature regarding approaches for spherical robots as well as 3D scanning mechanisms. 

\subsection{Spherical Robots}
\label{sec:stateOfTheArt:sphericalrobots}
An early idea of an self-driving sphere has been introduced by J.L. Tate in 1893 who claimed the patent  508 and 558 in  the U.S. for a sphere, driven by an inner moving counterweight, which got its torque from an spring. This idea of an actuator attached to an counterweight pointing to the bottom and therefore the torque being transferred to the sphere and moving it, is still a wide spread approach for spherical robots.

In \cite{soa1} a basic motion controll system for the BYQ-III is introduced. The BYQ-III has a mass of 25kg and a diameter of 600mm and its driving mechnaism has been proposed in \cite{soa2} by S.Hanxu, X,Aiping, J.Qingxuan and W.Liangqing. It contains  a counterweight pendulum, four gyros providing movement for two axes and one IMU mounted on the gyro case. There is no extra payload or Sensor, nor would there be space for a centered measurement unit due to the centered counterweight. Therefore the counterweight leads to a steady movement no relying on acceleration but on velocity of the actuators and therefore providing continuous speed.

A second spherical robot with its driving mechanism relaying on inner counterweight is preented in \cite{soa3}. This was designed for movement on water-surface and therefore having orthogonal to the movement mounted fins on the shell. Two actuators attached to the shell and the inner counterweight provide movement around one axis. In contrast to the BYQ-III a middle-centered  sensor would be possible, but this would have no movement relative to the surface, as it would be part of the relatively non moving inner counterweight. It also has steady, well controllable movement. 
The sphere presented in \cite{soa4} \cite{soa5} provides an solution for a driving system which does not relay on a moving inner counterweight but uses internal reaction wheels to provide torque. This leads to theoretically having middle-centered space which is rolling an not steady steady to the environment.  But the prototype provided by Vijay Muralidharan shows less controllability then counterweight driven spheres. Also now it is driven by acceleration and not velocity which leads to limited movement.

A third approach for spherical robots relays on an internal unit which drives in the sphere. A Design and control approach is provided in \cite{soa6} where an four-wheeled vehicle moves in the sphere to force it rolling by moving the center of mass in the desired direction. This technical solution is capable of a nearly maximum size of possible payload in relation to the overall-size, but also does not  provide without further mechanics a rotation of the senser which would be needed for 3-dimensional laser scanning.Also this provides just like the counterweight driven approach a good controlability, it is obviously not as stable regarding external perturbations or forces, due to the missing fixed connection to the shell. This would make it not suitable for mission with extreme forces and unknown starting conditions like missions involving a rocket launch or an drop to the starting point, which would lead to harsh movements of the in the worst case a start with the unit being rotated 180 degree and therefore not being able to bring torque to the sphere. This would also happen if the sphere was stuck to to the environment and therefore the inner unit trying to perform a whole revolution of the sphere, which leads to a supine position.

Overcoming this shortage of the inner unit just relaying on gravity to apply its force to the shell, \cite{soa7} introduced the approach of a rod, expanded by a spring to the maximum possible size and having a wheel on one side. The wheel generates the movement by again moving the center of weight. But now the non-reversible supine position does not exist anymore, because even with the wheel being at the top, it still is pressed to the shell by the spring and therefore being able of maintaining its movement. Again this approach does not provide spin of a centered  placed sensor without further contraptions.
