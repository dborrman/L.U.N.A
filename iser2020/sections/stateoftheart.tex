\section{State of the Art}
\label{sec:stateOfTheArt}
As evolution of RADLER, a hand-driven radial scanner device, a self-driving spherical approach was chosen to ensure robustness and autonomy.
The following presents different approaches to spherical robots.

An early idea of an self-driving sphere has been introduced by J.L. Tate in 1893 \cite{tate1893sphere} for a sphere, driven by an inner moving counterweight, which got its torque from an spring.
This idea of an actuator attached to an counterweight pointing to the bottom and therefore the torque being transferred to the sphere and moving it, is still a widespread approach for spherical robots.

In~\cite{soa1} a basic motion control system for the BYQ-III is introduced.
The BYQ-III has a mass of \unit[25]{kg} and a diameter of \unit[600]{mm} and its driving mechanism has been proposed in~\cite{soa2} by Hanxu et al.
It contains  a counterweight pendulum, four gyro actuators, providing movement for two axes and one IMU mounted on the case of the gyro.
There is no extra payload or sensor, nor would there be space for a centered measurement unit due to the centered counterweight.
Therefore the counterweight leads to a steady movement, not relying on acceleration but on velocity of the actuators and therefore providing continuous speed.

A second spherical robot with its driving mechanism relaying on inner counterweight is presented in \cite{soa3}.
This robot was designed for movement on a water surface and therefore has fins mounted to the shell orthogonal to the movement.
Two actuators attached to the shell and the inner counterweight provide movement around one axis.
In contrast to the BYQ-III a middle-centered sensor would be possible, but this would have no movement relative to the surface, as it would be part of the relatively non moving inner counterweight.
It also has steady, well controllable movement. 
The spheres presented in~\cite{soa4} and~\cite{soa5} provide a solution for a driving system which does not rely on a moving inner counterweight but uses internal reaction wheels to provide torque.
This leads to theoretically having middle-centered space available, which would be rotating with respect to the surface.
However, the prototype provided by Muralidharan et al.~\cite{soa4} shows less controllability then counterweight driven spheres. Furthermore, this robot is driven by acceleration and not velocity which leads to limited movement capabilities.

A third approach for spherical robots relies on an internal unit which drives inside the sphere.
A design and control approach is provided in~\cite{soa6} where a four-wheeled vehicle moves in the sphere to initiate rolling by moving the center of mass in the desired direction.
This technical solution is capable of a nearly maximum size of possible payload in relation to the overall-size, but also does not provide a rotation of the sensor is needed for 3D laser scanning.
Additionally, this provides good controllability, similar to the the counterweight-driven approach.
It is not stable regarding external perturbations or forces, as there is no fixed connection to the shell.
This makes it not suitable for missions with extreme forces and unknown starting conditions like space applications involving a rocket launch or a hard landing to get to the starting point.
In a worst case scenario this leads to a start with the unit being rotated by 180 degrees and therefore not being able to bring torque to the sphere.
If the sphere is stuck a similar situation occurs.
Then the inner unit tries to perform a whole revolution (``looping'') inside the sphere, which might result in a supine position.

Halme et al.~\cite{soa7} overcome this shortage of the inner unit with just relying on gravity to ensure contact to the shell.
They introduce the approach of a rod, expanded by a spring to the maximum size and having a wheel on one side of the rod and the original unit on the opposite site.
The wheel generates the motion by moving the center of mass, so the non-reversible supine position does not exist anymore.
Even if the center of mass is at the top, the vehicle is still pressed to the shell by the spring and therefore is capable of maintaining its movement.
This approach does not rotate a sensor placed in the center without further contraptions.

Inspired by the seminal Cubli  \cite{cubliIROS12}, we started designing a spherical robot based on a conservation of momentum (COAM) drive. 